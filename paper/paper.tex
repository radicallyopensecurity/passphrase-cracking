\documentclass{article}

\usepackage{float}
\usepackage[utf8]{inputenc}
\usepackage{hyperref}
\usepackage[hyperref=true, natbib=true, style=numeric, backend=bibtex]{biblatex}
\usepackage{graphicx}
\bibliography{references.bib}

\setlength{\parindent}{0pt}
\setlength{\parskip}{\baselineskip}

\begin{document}

\title{Probabilistic Passphrase Cracking}
\date{\today{}}
\author{Luc Gommans \\
	Radically Open Security
}

\maketitle

\section*{Abstract}

% TODO: write abstract


\section{Introduction}

% TODO: check section

Requirements for secure passwords are getting more and more stringent to stay
ahead of the increase in computational capacity of commonly accessible systems.
Using passphrases, users can approach the problem differently: form a string of
random, common words, instead of a complex sequence of characters. The most
famous example is `correct horse battery staple': four words with no coherence,
but many people in the field can recall this phrase without ever trying to
memorize it\cite{own}. Studies also show that users have fewer difficulties
when memorizing a passphrase than when memorizing equally-strong random
passwords\cite{behavioral-analysis}\cite{pwd-memorability}.

Password cracking tools are widely available, which help us understand the
strength of them. For passphrases, there is no commonly used approach using a
single tool or dataset, probably because there are very few of them.

In this work, we will attempt to create a program which generates likely
passphrases. It can be used in combination with traditional tools such as
Hashcat or John the Ripper. We will use a probabilistic method to create
realistic sentences and use rulesets of the traditional tool to do additional
manipulation such as uppercasing the first word, removing spaces, and other
operations that can be done using the optimized code of that tool.

In the next section, we will explore the state of the field. Section
\ref{sec:problemstatement}, Problem Statement, describes our expected
contribution. In order to obtain a set of passphrases to work with, we
reproduced Labrande's previous work in section \ref{sec:dataset}. Next, we
explore possible probabilistic methods in section \ref{sec:methods-comparison}.
Details about the implementation of the chosen method are set out in section
\ref{sec:implementation} and an analysis of its performance in section
\ref{sec:eval}. Finally, we describe our conclusions in section
\ref{sec:conclusions} and possible future work in section \ref{sec:futurework}.


\section{Related Work}

Sparell and Simovits\cite{sparell-simovits} created a passphrase cracking
program by using Markov chains on a word level. Markov chains are used in many
applications, such as next word prediction in keyboards on smartphones.
Similarly, one can use them to approximate the next most likely word in a
passphrase given a few starting words (which could be a dictionary walk). In
Sparell and Simovits' work, the LinkedIn hashdump of 2012 was used as one of
the test sets. This contains mostly passwords (not phrases), but is still
frequently used due to its large size of 64 million hashes. Of these, 21 000
phrases were cracked of between 10 and 20 characters in length.

In previous work, we\cite{own} generated passphrases from quotes of various
well-known persons and song lyrics. Using this method on the LinkedIn set, we
recovered 92 000 passphrases between 15 and 67 characters. A relatively large
amount of time was spent on collecting this data, which we think could be
optimized in future work. We also looked into whether users incorporate
personal data, such as lyrics specifically from a band they enjoy, but we have
not found this to be the common case.

Labrande\cite{crackmeimfamous} attempted to crack hashes from the Korelogic
dump (139 million entries) used a compilation of different public sources.
Around 4.3 million passwords were cracked, of which 200 000 are 16 characters
or longer. Public sources include quotes from WikiQuote and text written in
bold and italics on Wikipedia.

Bonneau and Shutova\cite{payphrase-properties} examined ,,patterns of human
choice in a passphrase-level authentication system deployed by Amazon''. They
,,tested the availability of [over] 100,000 possible phrases at Amazon's
registration page, which prohibits using any phrase already registered by
another user.'' This system allows passphrases as short as two words and
offered no explanation to users regarding what might be secure. Its interface
merely asks for a "phrase" (which is generally a string of {\it logical} words)
and even suggests two-word phrases to use. The conclusion drawn by this paper
is not surprising, given these circumstances: ,,users aren’t able to choose
phrases made of completely random words, but are influenced by the probability
of a phrase occurring in natural language. Examining the surprisingly weak
distribution of phrases in natural language, we can conclude that even 4-word
phrases probably provide less than 30 bits of security which is insufficient
against offline attack.'' We consider this low entropy estimation to be
specific for Amazon PayPhrase.

Weir\cite{weir} examined probabilistic methods for password cracking,
specifically Markov chains and probabilistic context-free grammar. Such
techniques might also be applied to passphrases. Besides examining different
methods, they also created an implementation using probabilistic context-free
grammar. This method cracked more passwords than John the Ripper (JtR), when
JtR is used in dictionary mode with the default word-mangling rules. It appears
that this method is well-suited for password cracking.


\section{Problem Statement}\label{sec:problemstatement}

Our main research question is: {\it How can software efficiently generate
likely passphrases, to be used in passphrase cracking?}

To achieve this, we will use a probabilistic method which generates phrases.
Which specific method to use, is part of this research. Probabilistic methods
can account for linguistic properties of phrases. It also allows users to
download a model, rather than a large list of (people's private) passphrases to
use directly. See section \ref{sec:methods-comparison} for a comparison of
possible methods.

Probabilistic methods require a dataset to be trained on. Because very few
systems use passphrases exclusively, there are no lists of commonly used
passphrases like there are for passwords. We will attempt to reproduce
Labrande's research\cite{crackmeimfamous} in order to create such a set. The
result of this attempt is described in section \ref{sec:dataset}.

We have also downloaded a large plaintext password dump, specifically the
"breach compilation" of 1.4 billion passwords, but this proved to mainly many
very insecure passwords and did not contain more than a few passphrases.


\section{Passphrase Dataset}\label{sec:dataset}

In order to train a probabilistic algorithm, a dataset is needed to train with.
Labrande\cite{crackmeimfamous} described cracking over 4 million entries from
the Korelogic\cite{korelogic} MD5 dataset, 200 000 of which are 16 or more
characters. This was done by downloading all articles of Wikipedia and
Wikiquote in various languages, plus a few other, smaller sources. Page titles,
list item entries, bold text, and italic text were used as phrases. Finally,
they used rules in John the Ripper to test variants of each phrase. They also
used rules which mangle the phrase into a password, for example by taking only
the first letter of each word in the phrase.

Because we are mainly interested in English passphrases, we downloaded only the
English Wikipedia and Wikiquote articles. For a set of hashes, we used the
LinkedIn as well as the Korelogic dump. This also provides a way of comparing
results with Sparell and Simovits'\cite{sparell-simovits} Markov chain-based
method, which they tested on the LinkedIn set. We processed the list in a
similar manner: extracting text in italics and bold, page titles, and list
items on Wikiquote. We used only a few phrase mangling rules, which are
described in section \ref{sec:implementation}. Since we are, at this point,
strictly interested in passphrases, we did not include many of the more
elaborate rules: rules such as taking the first letter of each word, or
appending random numbers, were not used. These defeat the purpose of a
passphrase, which is to be a set of common words with no punctuation or other
random tweaks to remember.

Labrande described the cracking process took a few weeks on "an outdated laptop
with a 2GHz processor", processing around 65 million unique entries in the
phrase dictionary. We extracted around 26 million unique entries, for a total
size of 690MiB. The difference in number of entries can be explained by the
fact that we did not download other languages. Processing these 38 million
entries, including the mangling rules, took around 10 minutes using Hashcat on
an Intel Core i7-3630QM.

From the LinkedIn set, this yielded 1.3 million results, of which 13 361 of 16
characters or longer. 26 345 cracked passwords include at least one space,
meaning they are probably passphrase-like even if they are shorter than sixteen
characters. In previous work\cite{own} we found that only about half the users
left spaces intact in passphrases, a number that is probably lower in a set
where users were not specifically queried about passphrases.

From the Korelogic MD5 set, we cracked 2.3 million hashes in 20 minutes, using
the same list of phrases and mangling rules. 146 742 were 16 characters or
longer and 311 815 included a space.


\section{Probabilistic Method Selection}\label{sec:methods-comparison}

Possible probabilistic approaches are Markov chains, weighted context-free
grammar and n-grams. Such methods have been applied to password cracking in
the past, and might be applied to passphrases as well. Markov chains is one
example which has also been applied to passphrases\cite{sparell-simovits} with
good results.

As far as we have been able to find, probabilistic context-free grammar has not
been applied to passphrase cracking. It proved to be a viable method in
Weir's\cite{weir} password cracking work. For passphrases, it could be adapted
to operate on whole words rather than individual characters. Based on texts and
previously cracked passphrases, rules could be derived which indicate for
example how likely it is for a noun to follow a verb.

In order to test whether this is a viable method, we attempted to classify the
type of word in cracked passphrases with a python program which used the
WordNet 3.0 database\cite{wordnet}. Because not all passphrases separate words
by spaces, the program would try to find words in phrases. The string
`correcthorsebattery' would result in the identified words `correct', `horse'
and `battery'. However, the WordNet database does not contain any conjugations
or plurals. In the string `hadadrink', it would not find the word `had' (it
only contains `to have'). The first known word would be `dad', after which it
would recognize `ink' as the next word. Incorrect identification of words
following an incorrect identification was extremely common. Particularly in
long words, it might recognize many small words incorrectly.

We evaluated the possibility of adding conjugations and plurals in some way, at
least for the most common words, but this quickly proved to be too large a task
to fit within the scope of this project.

While analysing the dataset, we saw that it contained many geographical names,
brands and person's names. Additionally, many of the phrases have the ring of a
catchphrase to them or form a common saying, for example "eye for an eye" or
"give me a break". While a probabilistic context-free grammar model might be
able to approximate those using large lists of names (and other words like
verbs, adjectives, etc.), n-grams capture parts of phrases and will inherently
become biased towards common phrases and constructions.

N-grams automatically include names and common constructions because they are
subsets of sentences, and implementations can be very fast because the
algorithm is fairly simple. The downside is that it has no memory: if it learns
`there are' and `are there' are common constructions, it will generate `are
there are', `there are there'. Markov chains have the same issue, but
context-free grammar does maintain more context (despite its name). We
chose to implement a proof of concept using n-grams because it is a novel
technique to generate passphrases and it seems fit for the purpose.


\section{Implementation}\label{sec:implementation}

,,[An] {\it n-gram} is a contiguous sequence of {\it n} items from a given
sample of text or speech.''\cite{wiki-ngram} In our case, each item is a word.
N-grams of the order two and three and also called {\it bigrams} and {\it
trigrams}, respectively.

We implemented a program in Python which splits a text on spaces and counts
each possible subsentence of {\it n} words. For example, in the sentence `lay
of the land', trigrams are `lay of the' and `of the land'. The number of times
an n-gram occurs is counted. Another Python program uses these n-grams to
generate phrases, using the most frequently occurring n-grams first. The source
code for these programs can be found in our git repository\cite{git}. The
output of the second program can be directly used in tools such as Hashcat.

The generator can make arbitrarily long phrases, as long as it can find a
possible continuation in the n-gram file (which is usually infinite because of
repetitions). The maximum length is therefore configurable. A logical value is
around eight to ten words: this is not long enough to output too many
repetitions while still generating fairly long phrases.

A bigrams file was initially generated from the passphrases we cracked in
section \ref{sec:dataset}. This yielded only 761 bigrams and 331 trigrams
because most passphrases are stripped of spaces. We attempted to use the
program which we used in section \ref{sec:methods-comparison} to find
concatenated words within phrases, splitting `batterystaple' into `battery' and
`staple'. Again we found that the identification errors were too great to be of
any use, even when using an additional dictionary\cite{wamerican} which
contains conjugations. (This additional dictionary was not used initially
because it does not have word type classifications.)

In order to supplement the system with more possible sentence continuations, we
used English Wikipedia articles. This yielded around 29 million bigrams and 47
million trigrams. We combined them with the n-grams from the cracked
passphrases, prioritizing the latter. This way, n-grams from Wikipedia are only
used supplementarily and n-grams from previously cracked passphrases are
tried first.


\section{Evaluation}\label{sec:eval}




\section{Conclusions}\label{sec:conclusions}

% TODO: compare Labrande's method with SparellSimovits' method (1.3M results of ours vs 20.5k of SS,
%   or 13k (ours) of 16+ chars vs 384 of SS.


\section{Future work}\label{sec:futurework}




\printbibliography

\end{document}

