\documentclass{article}

\usepackage{float}
\usepackage[utf8]{inputenc}
\usepackage{hyperref}
\usepackage[hyperref=true, natbib=true, style=numeric, backend=bibtex]{biblatex}
\usepackage{graphicx}
\bibliography{references.bib}

\begin{document}

\title{Probabilistic Passphrase Cracking}
\date{\today{}}
\author{Luc Gommans \\
	Radically Open Security
}

\maketitle

\section*{Abstract}

Passphrases are an alternative to passwords that are supposed to be easier to
remember for users. For passwords, cracking tools are widely available, helping
us understand their strengths and weaknesses. For passphrase cracking, there is
currently no common approach which is universally seen as the most effective
way.

In this work, we attempted to find and implement the most effective method. We
compared Markov chains, n-grams and a hybrid dictionary attack. The latter was
found to be most effective, both in terms of efficiency and the number of
results. Excluding the time required to build the dictionary from English
Wikipedia and Wikiquote articles, it took only 10 minutes to crack over 2.3
million unique passphrases by using a hybrid dictionary attack with hashcat.
Future users could download such a dictionary without having to compile it
manually from various sources.

\setlength{\parindent}{0pt}
\setlength{\parskip}{\baselineskip}

\newpage

\section{Introduction}

Requirements for secure passwords are getting more and more stringent to stay
ahead of the increase in computational capacity of commonly accessible systems.
Using passphrases, users can approach the problem differently: instead of a
complex sequence of characters, form a string of random, common words. The most
famous example is `correct horse battery staple': four words with no coherence,
but many people in the field can recall this phrase without ever trying to
memorize it\cite{own}. Studies also show that users have fewer difficulties
when memorizing a passphrase than when memorizing equally-strong random
passwords\cite{behavioral-analysis}\cite{pwd-memorability}.

Password cracking tools, which help us understand the strength (or weakness) of
passwords, are widely available. For passphrases, there is no commonly used
approach using a single tool or dataset, probably because there are very few of
them.

In the next section, we will explore the state of the field. Section
\ref{sec:problemstatement}, Problem Statement, describes our expected
contribution. In order to obtain a set of passphrases to work with, we
reproduced Labrande's previous work in section \ref{sec:dataset}. Next, we
explore possible probabilistic methods in section \ref{sec:methods-comparison}.
Details about the implementation of the chosen method are set out in section
\ref{sec:implementation} and an analysis of its performance in section
\ref{sec:eval}. Finally, we describe our conclusions in section
\ref{sec:conclusions} and possible future work in section \ref{sec:futurework}.


\section{Related Work}

Sparell and Simovits\cite{sparell-simovits} created a passphrase cracking
program by using Markov chains on a word level. Markov chains are used in many
applications, such as next word prediction in keyboards on smartphones.
Similarly, one can use them to approximate the next most likely word in a
passphrase given a few starting words (which could be a dictionary walk). In
Sparell and Simovits' work, the LinkedIn hash dump of 2012 was used as one of
the test sets. This contains mostly passwords (not phrases), but is still
frequently used due to its large size of 64 million hashes. Of these, 21 000
phrases of between 10 and 20 characters in length were cracked.

In previous work, we\cite{own} generated passphrases from quotes of various
well-known persons and song lyrics. Using this method on the LinkedIn set, we
recovered 92 000 passphrases between 15 and 67 characters. A relatively large
amount of time was spent on collecting this data, which we think could be
optimized in future work. We also looked into whether users incorporate
personal data, such as lyrics specifically from a band they enjoy, but we have
not found this to be the common case.

Labrande\cite{crackmeimfamous} attempted to crack hashes from the Korelogic
dump (139 million MD5 entries) using a compilation of different public sources.
Around 4.3 million passwords were cracked, of which 200 000 are 16 characters
or longer. Public sources include quotes from Wikiquote and text written in
bold and italics on Wikipedia.

Bonneau and Shutova\cite{payphrase-properties} examined ,,patterns of human
choice in a passphrase-level authentication system deployed by Amazon''. They
,,tested the availability of [over] 100,000 possible phrases at Amazon's
registration page, which prohibits using any phrase already registered by
another user.'' This system allows passphrases as short as two words and
offered no explanation to users regarding what might be secure. Its interface
merely asks for a `phrase' (which is generally a string of {\it logical} words)
and even suggests two-word phrases to use. The conclusion drawn by this paper
is not surprising, given these circumstances: ,,users aren't able to choose
phrases made of completely random words, but are influenced by the probability
of a phrase occurring in natural language. Examining the surprisingly weak
distribution of phrases in natural language, we can conclude that even 4-word
phrases probably provide less than 30 bits of security, which is insufficient
against offline attack.'' We consider this low entropy estimation to be
specific for Amazon PayPhrase.

Weir\cite{weir} examined probabilistic methods for password cracking,
specifically Markov chains and probabilistic context-free grammar. Such
techniques might also be applied to passphrases. Besides examining different
methods, they also created an implementation using probabilistic context-free
grammar. This method cracked more passwords than John the Ripper in dictionary
mode with the default word-mangling rules. It appears that this method is
well-suited for password cracking.


\section{Problem Statement}\label{sec:problemstatement}

Our main research question is: {\it How can software efficiently generate
likely passphrases, to be used in passphrase cracking?}

To achieve this, we will use a probabilistic method which generates realistic
phrases. Which specific method to use, is part of this research. We want to use
a probabilistic method because they can account for linguistic properties of
phrases. It also allows users to download a model, rather than a large list of
phrases to use directly. In section \ref{sec:methods-comparison} we compare
different methods.

Our implementation will be used in combination with traditional password
cracking tools such as Hashcat or John the Ripper, because those have fast
implementations of hash functions and other helpful features. One such feature
is rule sets, which we will use for generating variants of phrases to be tried.
These rule languages can perform simple manipulations such as lowercasing each
word, removing spaces, or appending an exclamation mark.

Probabilistic methods require a dataset to be trained on. Because very few
systems use passphrases exclusively, there are no lists of commonly used
passphrases like there are for passwords. We will attempt to reproduce
Labrande's research\cite{crackmeimfamous} in order to create such a set. The
result of this attempt is described in the next section.


\section{Passphrase Dataset}\label{sec:dataset}

In order to train a probabilistic algorithm, a dataset is needed to train on.
Labrande described cracking over 4 million entries from the
Korelogic\cite{korelogic} MD5 dataset, 200 000 of which are 16 or more
characters. This was done by downloading all articles of Wikipedia and
Wikiquote in various languages, plus a few other, smaller sources. Page titles,
list item entries, bold text, and italic text were used as phrases. Finally,
they used rules in John the Ripper to test variants of each phrase. They also
used rules that mangle the phrase into a password, for example by taking only
the first letter of each word in the phrase.

We classify this attack as a hybrid dictionary attack, because of the
similarity to traditional hybrid dictionary attacks on passwords. In suchs
attacks, a large list of words is used together with mangling rules to generate
possible permutations. In Labrande's variant, phrases are used instead of
words, using a smart method of selecting likely phrases.

Since we are mainly interested in English passphrases, we downloaded only the
English Wikipedia and Wikiquote articles. For a set of hashes, we used the
LinkedIn as well as the Korelogic dump. This also provides a way of comparing
results with Sparell and Simovits'\cite{sparell-simovits} Markov chain-based
method, which they tested on the LinkedIn set. We processed the data in a
similar manner: extracting text in italics and bold, page titles, and list
items on Wikiquote. We used only a few phrase mangling rules, which are
described in section \ref{sec:implementation}. Since we are, at this point,
only interested in passphrases, we did not include many of the more elaborate
rules: rules such as taking the first letter of each word or appending random
numbers, were not used. Such random variations would defeat the purpose of a
passphrase for the user, which is to be a set of common words with no
punctuation or other random tweaks to remember.

The results of this effort are described in section \ref{sub:dataset-results}
(evaluation of the hybrid dictionary attack).

Additionally, a large plain text password dump was downloaded, specifically the
`breach compilation' of 1.4 billion passwords. This proved to mainly consist of
very insecure passwords and did not contain more than a few passphrases.

Note that the downloaded dumps did not contain usernames, and any cracked hash
values are kept confidential.


\section{Probabilistic Method Selection}\label{sec:methods-comparison}

Possible probabilistic approaches are Markov chains, weighted context-free
grammar and n-grams. Such methods have been applied to password cracking in
the past, and might be applied to passphrases as well. Markov chains is one
example which has also been applied to passphrases\cite{sparell-simovits} with
good results.

So far, probabilistic context-free grammar has not been applied to passphrase
cracking. It proved to be a viable method in Weir's\cite{weir} password
cracking work. For passphrases, it could be adapted to operate on whole words
rather than individual characters. Based on texts and previously cracked
passphrases, rules could be derived which indicate for example how likely it is
for a noun to follow a verb.

In order to test whether this is a viable method, we attempted to classify the
type of word in cracked passphrases with a python program which used the
WordNet 3.0 database\cite{wordnet}. Because not all passphrases separate words
by spaces, the program would try to find words in phrases. The string
`correcthorsebattery' would result in the identified words `correct', `horse'
and `battery'. However, the WordNet database does not contain any conjugations
or plurals. In the string `hadadrink', it would not find the word `had' (it
only contains `to have'). The first known word would be `dad', after which it
would recognize `ink' as the next word. Incorrect identification of words
following an incorrect identification was extremely common. Particularly in
long words, it might recognize many small words incorrectly.

We evaluated the possibility of adding conjugations and plurals in some way, at
least for the most common words, but this quickly proved to be too large a task
to fit within the scope of this project.

While analyzing the dataset, it was found that it contained many geographical
names, brands and people's names. Additionally, many of the phrases have the
ring of a catchphrase to them or form a common saying, for example "eye for an
eye" or "give me a break". While a probabilistic context-free grammar model
might be able to approximate those using large lists of names (and other words
like verbs, adjectives, etc.), n-grams capture parts of phrases and will
inherently become biased towards common phrases and constructions.

N-grams automatically include names and common constructions because they are
subsets of sentences, and implementations can be very fast due to the
simplicity of the algorithm. The downside is that it has no memory: if it
learns `there are' and `are there' are common constructions, it will generate
`are there are', `there are there'. Markov chains have the same issue, but
context-free grammar does maintain more context (despite its name). We chose to
implement a proof of concept using n-grams because it is a novel technique to
generate passphrases and it seems fit for the purpose.


\section{N-gram implementation}\label{sec:implementation}

,,[An] {\it n-gram} is a contiguous sequence of {\it n} items from a given
sample of text or speech.''\cite{wiki-ngram} In our case, each item is a word.
N-grams of the order two and three and also called {\it bigrams} and {\it
trigrams}, respectively.

A program was implemented in Python which splits a text on spaces and counts
each possible subsentence of {\it n} words. For example, in the sentence `lay
of the land', trigrams are `lay of the' and `of the land'. The number of times
an n-gram occurs is counted. Another Python program uses these n-grams to
generate phrases, using the most frequently occurring n-grams first. The source
code for these programs can be found in our git repository\cite{git}. The
output of the second program can be directly used in tools such as Hashcat.

The generator can make arbitrarily long phrases, as long as it can find a
possible continuation in the n-gram file (which is typically infinite because
of repetitions). The maximum length is therefore configurable. A logical value
is around eight words: this does not output too many illogical repetitions
while still generating fairly long phrases.

A bigrams file was initially generated from the passphrases that were cracked
in section \ref{sec:dataset}. This yielded only 761 bigrams and 331 trigrams
because most users stripped their passphrases of spaces. We attempted to use
the program mentioned in section \ref{sec:methods-comparison} to find
concatenated words within phrases, splitting e.g. `batterystaple' into
`battery' and `staple'. Again we found that the identification errors were too
great to be of any use, even when using an additional
dictionary\cite{wamerican} which contains conjugations. (This dictionary was
not used initially because it does not have word type classifications.)

In order to supplement the system with more possible sentence continuations, we
used English Wikipedia articles. This yielded around 29 million bigrams and 47
million trigrams. Those were combined with the n-grams from the cracked
passphrases, prioritizing the latter. This way, n-grams from Wikipedia are only
used supplementarily and n-grams from previously cracked passphrases are tried
first.

Finally, mangling rules were used in Hashcat to create simple variants for each
phrase. Hashcat can use multiple rule sets which are then combined. If, for
example, if in set 1 there are rules for both lowercasing and uppercasing a
phrase and in set 2 there are rules for removing spaces, the resulting phrases
for `A b' will be `a b', `A B', `ab' and `AB'. In our previous work, we learned
what kind of alterations users make to their phrases. Based on that information,
we created three sets of rules:

\begin{enumerate}
	\item \begin{itemize}
		\item use as-is
		\item lowercase all characters
		\item capitalize the first letter, lowercase the rest
		\item capitalize the first letter of each word, lowercase the rest (titlecase)
	\end{itemize}

	\item \begin{itemize}
		\item use as-is
		\item append a period
		\item append an exclamation mark
		\item append a question mark
	\end{itemize}

	\item \begin{itemize}
		\item use as-is
		\item remove all spaces
	\end{itemize}
\end{enumerate}

These rules create 32 variants for each input.

The system used for all tests is a laptop with an Intel i7-3630QM CPU and 8GiB
RAM.


\section{Evaluation}\label{sec:eval}

In this section the performance of different passphrase cracking methods is
evaluated, considering both the number of entries they manage to crack and
their efficiency.

Because we reproduced Labrande's hybrid dictionary attack (with slight
alterations) and cracked a set of hashes identical to those of Sparell and
Simovits'\cite{sparell-simovits}, the performance of all methods can be
compared directly.


\subsection{Hybrid dictionary}\label{sub:dataset-results}

Using the hybrid dictionary attack as described by Labrande (with slight
alterations), we cracked 2.3 million hashes from the Korelogic MD5 set. 146 742
were 16 characters or more. 311 815 included at least one space, meaning they
are probably passphrase-like even if they are shorter than sixteen characters.

This is only half as many as Labrande cracked: 4.2 million and 200 000 of 16
characters or more (there are no statistics on the number of cracked entries
with spaces). This difference could be explained by the fact that we were
strictly after English passphrases and did not use a large number of mangling
rules, and because we did not include other languages. Labrande used the
English, German, French, Italian and Spanish versions of Wikipedia and
Wikiquote.

Note that in previous work\cite{own}, we found that only about half the users
left spaces intact in passphrases. This number is probably even lower in a set
where users were not specifically queried about passphrases. The number of
passphrase-like entries probably more than twice as high as the number of
entries we found with spaces.

Of the LinkedIn set, we were able to crack 1.3 million entries, of which 13 361
are 16 characters or more and 26 345 cracked passwords include a space.

\subsubsection{Efficiency}

Labrande's hybrid dictionary attack took ,,a few weeks on an outdated laptop
with a 2GHz processor,'' using John the Ripper. When reproducing this attack,
the process took around 10 minutes in Hashcat. Our dictionary was smaller (26
million instead of 65 million entries) and used fewer mangling rules, which
could explain the difference. Despite those limits, we still cracked half as
many entries as Labrande. Despite the more than 2000-fold decrease in time, it
still yielded good results.

The storage requirements of this method are relatively large: our phrases file
(with 26 million entries) is 690MiB. Since this was generated from Wikipedia,
all Wikipedia and Wikiquote articles also had to be downloaded, which total
26GiB as download and 135GiB uncompressed.

Memory usage is not relevant for this attack, as each line can be read off of
disk and is immediately passed onto Hashcat.


\subsection{Markov chains}

Sparell and Simovits cracked 25 000 entries from the LinkedIn set, of which 384
are 16 characters or longer.

\subsubsection{Efficiency}

Sparell and Simovits described that generating phrases is the dominant part of
the time consumption for their cracking process. They therefore wrote the
output of their phrase generator to a file before inputting it into Hashcat,
classifying files by different sources and passphrase lengths. Generation times
differ from 16 minutes for a 2.6 million 10-character phrases file to 960 hours
(40 days) for 8.5 billion 20-character phrases. The numbers in terms of phrases
per second vary greatly, ranging from 22 500 to 2 459 phrases per second.

The size of the source files, which form the Markov model, is not described;
nor is the amount of RAM required for this method.


\subsection{N-grams}

% TODO: word choice (global): entries or records? Or something else?

Using the passphrase generator with trigrams, it was able to crack 835 562
entries of the Korelogic MD5 set, of which 32 911 are 16 characters or longer,
and 70 198 contain a space. Of the LinkedIn set, it was able to crack 482 014
entries, of which 3 782 are 16 characters or longer and 14 903 contain a space.

Using bigrams, it was able to crack 342 145 entries of the Korelogic set, of
which 9 148 are 16 characters or longer and 1 023 contain a space. Of the
LinkedIn set, it was able to crack 190 117 entries, of which 1 468 are 16
characters or longer and 1 449 contain a space.

Both attacks produced a great number of short results, such as `thefault' or
`wanderers'. After going through all phrases shorter than or equal to {\it n}
words, the number of results dropped significantly.

\subsubsection{Efficiency}

The n-gram phrase generator first loads the data, which takes a few minutes and
scales linearly with the n-gram file size. While running, it generates two to
four million phrases per second on a single CPU core (the median is 3.3 million
phrases per second). This saturated Hashcat's ability to process input: hashcat
processes around ten million SHA-1 hashes per second, but applies a few rules.
Since rules easily generate many more than three variants per phrase, the speed
of our program is sufficient despite running on only a single core and being
written in Python, an interpreted language.

Memory usage for the generation process depends on the number of n-grams
loaded: for 22 million trigrams (464MiB), it uses 5.3GiB of RAM. This is quite
intensive, but required for fast lookups. Bigrams are much more efficient:
because there are fewer combinations to be made, the input files are much
smaller. The complete bigrams file contains 2.9 million entries (47MiB) and
uses 800MiB RAM.

A file with trigrams of 464MiB is not much smaller than our Wikipedia and
Wikiquote phrase dictionary of 690MiB. Bigrams are more efficient, using 47MiB.

\subsection{Overview}

Table \ref{tab:comparison} gives an overview of the performance of the
different methods. The column headers `Dict-1' and `Dict-2' indicate the hybrid
dictionary attack, Labrande's and ours, respectively. The rows labeled `16+'
indicate the number of results which are 16 characters or longer.

Note that for our hybrid dictionary attack, the number of generated phrases per
second is estimated at 10 million. Contrarily to all other methods, this method
is IO-bound instead of CPU-bound. The amount of RAM required is considered
negligible, as it only requires IO buffers.

\begin{table}[H]
	\begin{tabular}{| l | r | r | r | r | }
	\hline                & {\bf Dict-1} & {\bf Dict-2} & {\bf Markov} & {\bf 3-grams} \\
	\hline LinkedIn       & n/a          &  1 300 000   &    25 000    &   482 014     \\
	\hline LinkedIn 16+   & n/a          &     13 361   &       384    &     3 782     \\
	\hline Korelogic      & 4 200 000    &  2 300 000   & n/a          &   835 562     \\
	\hline Korelogic 16+  &   200 000    &    146 742   & n/a          &    32 911     \\
	\hline Phrases/second & n/a          & 10 000 000   &    22 500    & 3 300 000     \\
	\hline RAM            & n/a          & negligible   & n/a          & 5.3GiB        \\
	\hline Storage        & n/a          &     690MiB   & n/a          & 464MiB        \\
	\hline
	\end{tabular}
	\caption{Performance overview}
	\label{tab:comparison}
\end{table}


\section{Conclusions}\label{sec:conclusions}

We have explored and compared different methods for passphrase cracking,
confirmed the results of Labrande's hybrid dictionary attack, and implemented a
new method based on n-grams. Our conclusion is that the most effective attack
at this time is the hybrid dictionary attack, even when considering the
trade-off between storage requirements and practicality versus the number of
results obtained. While the dictionary does need a relatively large amount of
storage, particularly when generating or updating it, it is extremely efficient
in terms of RAM and CPU.

N-grams are less effective than expected. The most effective part of the
cracking process were the phrases of {\it n} words or fewer, corresponding to
the order of the n-gram. This means that literal hits were much more common
than hits of combined n-grams, indirectly confirming again how powerful the
dictionary attack is. It also indicates that passphrases are generally not
normal, full sentences: despite training the model on both previously cracked
passphrases and all English Wikipedia articles, sentences generated from those
n-grams do not yield many results.

Markov chains, as described by Sparell and Simovits, yielded the fewest results
of the three methods we compared. Generating the passphrases was also slower
than by using n-grams or a dictionary, probably because the algorithm for
Markov chains is more complex.

The source code for our software is available on Github\cite{git} under GPL
version 3.0.


\section{Future work}\label{sec:futurework}

We have not been able to explore probabilistic context-free grammar fully due
to the difficulties in classifying words. Weir was successful in this with
passwords because he focused on a character level. For passphrases, which are
composed of words instead of characters, this method is more complex. It is
recommended to look into other, possibly commercial options for a word
classification database. It is also recommended to avoid using previously
cracked passphrases as training set: many are stripped of spaces, making it
very difficult to programmatically split words for classification.

Our n-gram attack can be improved in various ways. It might be more effective
to store which n-grams were the start of a sentence. In our version, it would
take an n-gram and append other, matching n-grams to complete the phrase.
However, if the starting n-gram is not a logical way to start a sentence, all
generated continuations are wasted. It would also be more memory- and
storage-efficient to use a language with less memory overhead, and to identify
words by tokens instead of indexing literal n-grams.

Finally, we are curious how neural networks would perform for this purpose.
Training a model is typically computationally intensive, but general purpose
frameworks are available which make it easy to run such computations on a GPU
and generate fairly large models on modest hardware.


\newpage
\printbibliography

\end{document}

