\documentclass{article}

\usepackage{float}
\usepackage[utf8]{inputenc}
\usepackage{hyperref}
\usepackage[hyperref=true, natbib=true, style=numeric, backend=bibtex]{biblatex}
\usepackage{graphicx}
\bibliography{references.bib}

\setlength{\parindent}{0pt}
\setlength{\parskip}{\baselineskip}

\begin{document}

\title{Practical Passphrase Cracking}
\date{\today{}}
\author{Luc Gommans \\
	Radically Open Security
}

\maketitle

\section{Introduction}

Most of user data is password protected, but the choice of these passwords is
largely left to users. Password recommendations propose using hard-to-remember
patterns in order to make it harder to guess, while in reality many of those
`random' tweaks (such as a number or exclamation mark at the end) are easily
iterable by a computer.

Using passphrases, users can approach the problem differently: form a string of
random, common words. The most famous example is `correct horse battery
staple': four words with no coherence, but many people in the field can recall
this phrase without ever trying to memorize it\cite{own}. Studies also show
that users have fewer difficulties when memorizing a passphrase than when
memorizing equally-strong random
passwords\cite{behavioral-analysis}\cite{pwd-memorability}.

Labrande\cite{crackmeimfamous} showed that at least 4 million in a set of 139
million passwords (2.9\%) consist of phrases from publicly available sources.
The exact percentage of passphrases used as passwords is not known, but this
shows they are used by a lof of users while they have been studied relatively
little.

Password cracking tools are widely available, which help us understand the
strength of them. For passphrases, there is no commonly used approach using a
single tool or dataset, probably because there are very few of them.

In this work, we will attempt to create a program which generates likely
passphrases. It can be used in combination with traditional tools such as
Hashcat or John the Ripper. We will use a probabilistic method to create
realistic sentences and use rulesets of the traditional tool to do additional
manipulation such as uppercasing the first word, removing spaces, and other
operations that can be done using the optimized code of that tool.


\section{Related Work}

Sparell and Simovits\cite{sparell-simovits} created a passphrase cracking
program by using Markov chains on a word level. Markov chains are used in many
applications, such as next word prediction in keyboards on smartphones.
Similarly, one can use them to approximate the next most likely word in a
passphrase given a few starting words (which could be a dictionary walk). In
Sparell and Simovits' work, the LinkedIn hashdump of 2012 was used as one of
the test sets. This contains mostly passwords (not phrases), but is still
frequently used due to its large size of 64 million hashes. Of these, 21 000
phrases were cracked of between 10 and 20 characters in length.

In previous work, we\cite{own} generated passphrases from quotes of various
well-known persons and song lyrics. Using this method on the LinkedIn set, we
recovered 92 000 passphrases between 15 and 67 characters. A relatively large
amount of time was spent on collecting this data, which we think could be
optimized in future work. We also looked into whether users incorporate
personal data, such as lyrics specifically from a band they enjoy, but we have
not found this to be the common case.

Labrande\cite{crackmeimfamous} attempted to crack hashes from the Korelogic
dump (139 million entries) used a compilation of different public sources.
Around 4.3 million passwords were cracked, of which 200 000 are 16 characters
or longer. Public sources include quotes from WikiQuote and text written in
bold and italics on Wikipedia.

To summarize, the problem of cracking passphrases is generally approached by
either using a large set of phrases directly, or by training a model on a large
set and using those to derive phrases which are likely to be logical.


\section{Research Questions}

Our main research question is: {\it How can software efficiently generate
likely passphrases, to be used in passphrase cracking?}

To achieve this, we will use a probabilistic method which generates phrases in
decreasing likelihood. An important factor in this design is the speed at which
it can do this: most methods are not designed to iteratively list all
possibilities in decreasing likelihood, which makes it difficult to do so using
methods such as context-free grammar (as described by
Weir\cite{probabilistic-thesis}, p. 84--85).

From this follow more questions to answer our main question:

\begin{enumerate}
	\item Which probabilistic methods can be used to generate output in decreasing order of likelihood?
	\item Which probabilistic method produces the most likely passphrases?
	\item How much data, both in terms of storage and random access memory, does this method require?
\end{enumerate}

Our questions focus on probabilistic methods because those do not require users
to download or even compile large lists of passphrases to use directly.
Instead, a model can be made available which contains probabilities for a
certain language.


\section{Scope}

The software to be created will interface with at least one common password
cracking tool, such as Hashcat or John the Ripper, and should be easy to use in
concert. It is considered out of scope to implement cracking in the software
itself, since that functionality can well be provided by other software.

The software will work for a general population and cannot be primed with facts
such as birth dates of specific users. In our previous work\cite{own}, we found
this to be not as effective as expected. It seems better to focus on the
general method in this limited-time project.


\section{Requirements}

For this project, a software prototype will be built which might use a
medium-sized dataset to operate. We estimate that a system with 4GiB RAM and
10GiB persistent storage is required. This is supplied by Radically Open
Security.

For software, we expect to use only free and open source products.


\section{Ethical Considerations}

In this investigation, we will need to evaluate the prototype. We will do this
using passwords from publicly available datasets. In all cases, the usernames
will be removed and the only information kept is the password (or a hash
thereof) and how frequently it occurred in the dataset.


\section{Planning}

\begin{tabular}{ | l | l | }
\hline {\bf Week 1} & Write proposal, choose probabilistic method \\
\hline {\bf Week 2} & Early prototype, interfacing with password cracking software \\
\hline {\bf Week 3} & Improving and evaluating prototype \\
\hline {\bf Week 4} & Finalise prototype and paper, prepare presentation \\
\hline
\end{tabular}


\printbibliography

\end{document}
